\section{Related work}
Automated self tracking is a popular topic recently. It drastically reduces the complexity of laborious tracking. In this section, we provide some background on Quantified Self movement. On the other hand, people strive to obtain self-knowledge but they may have problems while doing so. This section also talks about some valuable work that identifies a comprehensive list of problems on personal informatics. 
 
In the past, the methods of quantitative assessment on personal data tracking were relatively laborious and arcane. People needed to take measurements manually and record them in a log; people had to enter data into spreadsheets and perform operations using unfriendly software; people had to build graphs or charts to tease understanding out of the numbers. Now much of the data-gathering can be automated, and the record-keeping and analysis can be delegated to a host of simple Web apps which makes it possible to know oneself in a new way. 

Gary Wolf and Kevin Kelly created a blog called quantifiledself.com in 2007, which has become the repository for people to share self-tracking practices \cite{choe2014understanding}. The Quantified Self is an international collaboration of users and makers of self-tracking tools. This site is the base for a community of users who collaborate to share self-knowledge, tools and interests around the topic of self-tracking. Besides using the website to share information, it also has self show and tell meetings all over the world. Quantified-Selfers (Q-Selfers) are notable exceptions who diligently track many kinds of data about themselves. They are a diverse group of life hackers, data analysts, computer scientists, early adopters, health enthusiasts, productivity gurus, and patients \cite{choe2014understanding}. Additionally, there is a huge list broken down by categories over at The Quantified Self. So there is no shortage of devices and services out there to help us get data about ourselves with many more surely on the way. 

Personal informatics represents an interesting area of study in human-computer interaction. Trackers usually encounter a wide variety of issues while tracking their personal data. To address this, Li and his colleague conducted surveys and interviews with people who collect and reflect on personal information. They derived a stage-based model of personal informatics systems composed of five stages (preparation, collection, integration, reflection, and action) and identified barriers in each of the stages \cite{li2010stage}.

The stage-based model of personal informatics systems identifies problems across personal informatics tools through five stages. This model also improves the diagnosis, assessment, and prediction of problems in personal informatics systems. Identifying problems that people experience in collecting and making sense of personal information while using such system is significantly important in regard to designing and developing effective personal informatics.

In Frank Bentley’s Health Mashups \cite{bentley2013health} research paper he attempts to solve the modern dilemma that consumers are presented with abundant personal tracking information but do not have the necessary resources or motivation to do anything with this data. Bentley suggests an experiment involving the Health Mashups system which takes various self-tracked information both automatically and manually inputted, runs various statistical analysis and tests and outputs correlational statements in natural language to users. Through these statements, Bentley’s hopes were for users to have self-realizations and reflections, and with this awareness users would produce behavioral changes. The results of the experiment had a range of results including people who became more aware of their behavior and others who felt some statements were obvious or useless.