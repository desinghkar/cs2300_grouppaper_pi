\section{Introduction}
We have yet to encounter research on individuals conducting their own personal informatics experiments. In this paper we plan to fill in this gap in the field of personal tracking by providing an overview of twenty-one individuals' personal informatics experiments, each of whom were enrolled in the Human Computer Interaction Seminar at Brown University. Each experimenter was instructed to design their own experiment spanning a month and analyze their results upon completion. Every student had to choose at least one independent variable, two dependent variables, formulate at least two related hypotheses, and design an experiment to test their hypotheses. No one in the class was allowed to have the same set of variables and hypotheses as another classmate.  
 
All twenty-one students employed various methods to track their variables throughout their respective experiments.  The class was made up of twelve male and nine female students, representing every level of education offered at Brown University besides freshmen.  There were two sophomores, two juniors, four seniors, four Master’s students, six PhD students, and one unknown. (The one unknown is a result of student backgrounds being gained through a voluntary survey we sent out post-experiment completion. One member of the seminar did not fill out our survey.)
 
Every member of the class was given the same set of directions. These directions instructed that they could track any combination of independent and dependent variables, as long as there was a hypothesis that could be tested and that the data regarding the variables could be analyzed. Across the twenty-one studies, commonalities in the various aspects of the process of the experiment were observed, including dependent variables, confounding factors, and statistical results. While there were several distinct dependent variables, we explore questions such as: Why were some of the most common ones studied by almost everyone in the class? Why were so many users' results, post analysis, considered insufficient to make any definite claims? We believe that by examining these twenty-one reports on personalized experiments, we will be able to answer these questions and gain insight pertaining to the motivations behind tracking certain variables, factors necessary for successful personal-informatics tracking, and other valuable aspects of completing personal informatics experiments.