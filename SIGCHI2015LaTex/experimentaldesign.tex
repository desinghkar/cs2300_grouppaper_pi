\section{Experimental Design}

While performing various experiments, users had to choose an experimental design that would deliver the best results without introducing biases. During an experiment, especially one running for 28 days, experimental design is extremely important. During an experiment a user changes an independent variable in order to observe how it affects one or more dependent variables. In order to notice any and garner the most and best effects, the user must come up with a way to lay out how their experiment will run, thinking in advance of any factors that might affect their experiment and attempt to control for that. The 21 participants in this study chose various experimental designs but most can be broken down into two different designs. 
    The first design is involves randomization. Participants that chose this experimental design would flip a coin or find a random way to either perform or not perform their independent variable, such as running that day or not, or showering before sleep or not. This method limits environmental and temporal biases that might otherwise exist. Some examples of these biases include differences in behavior during weekdays versus weekends or even external circumstances such as having a busy day or week due to class projects, exams, homework, etc. Randomization ignores these factors because a user might be busy two days in a row but perhaps one day the coin flips ‘heads’ while the other day it flips ‘tails’ controlling for any personal biases. 
    The other common experimental design was the ABXX design and variations on that. These variations included ABA, AB, ABCD, ABA, ABAC. In this experimental design, users perform a certain behavior equally for a period of time (A) then switch to a different behavior for another period of time (B), and so on, each letter delineating a difference in the way they perform their independent variable. This experimental design fits some experiments better than say, randomization, because some effects might take longer to come into effect than others and so running it for a longer period of time (versus just one day) might affect the data. This experimental design also allows for participants to run experiments where they try a behavior, go to a different behavior (including the complete lack of that behavior), and finally return to the original behavior in order to analyze different effects. 
    Overall, users were careful in designing their experiments so that they could introduce the least amount of biases into their data. Unfortunately for some, especially those running the variations of ABXX experimental design, temporal biases were introduced especially due to the last week of the experiment coinciding with spring break whereas those running their experiment through randomization did not have as much of a problem with this confounding factor which we touch upon later in the paper. 
