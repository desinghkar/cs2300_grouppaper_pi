\section{Conclusion}
Upon reading all of the reports and examining the students' results in regards to our initial questions we reached several conclusions regarding our initial speculations.  
 
In terms of the dependent variables chosen, the most common variables the students chose were ones pertaining to integral aspects of their everyday life, i.e. sleep quality, productivity, heart rate and weight.  In terms of a college student's daily routine, one's physical health, sleeping habits and productivity in terms of schoolwork often have the largest effects on their day to day activities, and thus we speculate that the variables that are most important to the user's everyday life are the ones that people will be most interested in tracking. 
 
As far as the substantial number of results whose analyses rendered the data insufficient to make any claims, we attribute this to two things: First, because of the short period of time allotted for the experiment, there was an insufficient number of data points which hindered the observation of  statistically significant relationships. Secondly, there could have been improper tracking, both self-reported and automatic, on the part of the user or inaccuracies in the applications.  Many users mentioned, either in their reports or in the survey, that if the studies were done again, they would have either extended the duration of the experiment or scheduled their time better.  We speculate that if there had been more time, during which data could have been collected, more statistically significant relationships would have been observed. In regards to the tracking methods, we draw upon both our personal experiences, the findings we read in the reports, and statements made in the survey that we collected to arrive at the claim that passive tracking is the most effective. Many users reported that when it came to self reporting, they would often forget to log information at the same time each day.  The methods of tracking deemed the most reliable were ones that were passive or automatic and required minimal alteration to the experimenter's habits. 
 
On the matter of future work, we ask ourselves if this is a legitimate form of personal informatics experimentation. We believe that with some adjusting of the method employed by the Human Computer Interaction seminar members, that this can still produce legitimate findings.  Issues revealed throughout the course of the study pertained to time constraints, scheduling conflicts, and inconsistent tracking, and not because the students did not know themselves well enough.  We think that conducting a personally tailored study, especially in regards to matters such as mood or productivity that are relative to one's own past experiences, it is acceptable to design one's own experiments and accompanying hypotheses.  There does, however, seem to be a need for an extension on the time during which data is collected.  The issue of insufficient numbers of data points along with the fact that some effects or relationships may take longer to be observed in full, led us to conclude that longer amounts of time would be desirable. Unlike many studies done in the past such as \textit{Health Mashups} that led some users to unwanted information, this way of performing personal informatics is both efficient and more prone to creating behavioral changes because it asks the user to design their own hypothesis and test variables that they care about. It is completely tailored to the user and we believe with a few fixes to the assignment such as an experiment that runs for a longer period of time, a more thought out experimental design, and with a refined background in statistic, this method of running personal informatics can be tailored to give reliable, statistically significant insights pertaining to an individual.