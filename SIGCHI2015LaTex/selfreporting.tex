\section{Self Reporting}
Many students used self tracking in their experimentation. Different variables were tracked either using an app or a self-defined questionnaire. Mobile and desktop apps were used to track subjective variables such as mood. Apps with predefined scales for measuring variables were the preferred method of tracking subjective data. These apps sent scheduled reminders to track variables in the app, which helped many of the students consistently track the stages of their experiments. 

Several students created their own questionnaires to help them track their dependent variables. 
Participants felt that answering the same questions lessened the bias in their self reporting since they could record the same data in the same manner and do so with the same rating scale in mind. One participant came up with a series of questions and categorizations to classify the content and type of her dreams. Most of her questionnaire was very detailed and became less so towards the end of it. She mentioned that tracking her dreams might influence her answers later as she would get better at remembering her dreams and such, but that she tried to stay consistent in her ratings to have accurate data. Another self reported mood. He rated his mood three times a day (on waking, at noon, and before bed) on a scale of 1 - 5. 

Even though the students tried to lessen the bias of their self reporting, they felt that self reporting was the the most biased and error prone way to track in comparison to other devices and applications. A problem students had with self reporting was that other external factors would sometimes influence their ratings for mood, well-being, or tiredness and these events would not be catalogued, creating only a partial picture of what is really going on in their life. Participants noted these events may have influenced their results but felt they were integral to the study since they were studying themselves [??? Is this correct???]

Even though the use of apps and devices appears to be the more mainstream and preferred way of tracking dependent variables, some types of data are too subjective to track automatically such as mood, tiredness and well being. These variables are best cataloged through self reporting. Some of our participants created their own method to accurately track the variables they were interested in to increase the accuracy of the data collected. Self reporting has tremendous potential in areas where apps cannot accurately gauge what kind of data users want to collect. It could also be useful for self trackers who are a lot older. Since this demographic of people typically do not use apps or modern technology, manual tracking could be a helpful way for these people to track their personal information. 
